% Created 2021-03-17 mié 01:12
% Intended LaTeX compiler: pdflatex
\documentclass[letterpaper]{article}
\usepackage[utf8]{inputenc}
\usepackage[T1]{fontenc}
\usepackage{graphicx}
\usepackage{grffile}
\usepackage{longtable}
\usepackage{wrapfig}
\usepackage{rotating}
\usepackage[normalem]{ulem}
\usepackage{amsmath}
\usepackage{textcomp}
\usepackage{amssymb}
\usepackage{capt-of}
\usepackage{hyperref}
\usepackage[, spanish]{babel}
\author{Cardoso Moreno Marco Antonio}
\date{\today}
\title{Tarea 1 - Demostraciones de las propiedades de los órdenes asintóticos}
\hypersetup{
 pdfauthor={Cardoso Moreno Marco Antonio},
 pdftitle={Tarea 1 - Demostraciones de las propiedades de los órdenes asintóticos},
 pdfkeywords={},
 pdfsubject={},
 pdfcreator={Emacs 27.1 (Org mode 9.5)}, 
 pdflang={Spanish}}
\begin{document}

\maketitle

\section{Demostrar:}
\label{sec:orgbe567fb}
\subsection{\(f(n)\) es \(\Theta(g(n))\) si y sólo si es \(O(g(n))\) y es \(\Omega(g(n))\)}
\label{sec:org5cd6a49}
\begin{itemize}
\item Si \(f(n)\) es \(\Theta(g(n))\), existen las constantes \(c_f', c_f\) y \(n_f\) tal
que
\begin{itemize}
\item \(c_f' g(n) \leq f(n) \leq c_f g(n)\) para toda \(n \geq n_f\).
\end{itemize}
\end{itemize}

\noindent
Si esta desigualdad la dividimos en dos, las nuevas desigualdades quedan:
$$
c_f' g(n) \leq f(n) \quad \mbox{y} \quad f(n) \leq c_f g(n)
    \quad \mbox{para toda} \quad n \geq n_f
$$

\noindent
Se observa que estas desigualdades son las definiciones tanto de
\(f(n) = O(g(n))\) (derecha), como de \(f(n) = \Omega(g(n))\) (izquierda), con lo
que se comprueba el enunciado.
\subsection{Transitividad}
\label{sec:org3ddb54e}
\subsubsection{Si \(f(n)\) es \(O(g(n))\) y \(g(n\)) es \(O(h(n))\) entonces \(f(n)\) es \(O(h(n))\)}
\label{sec:org350b076}
Como se comentó en los videos, debemos partir de que el antecedente es cierto,
por lo tanto, sabemos que:
\begin{itemize}
\item Si \(f(n)\) es \(O(g(n))\), existen dos constantes \(n_f\) y \(c_f\) tal que
\begin{itemize}
\item para toda \(n \geq n_f\), \(c_f g(n) \geq f(n)\)
\end{itemize}
\item Si \(g(n)\) es \(O(h(n))\), existen dos constantes \(n_g\) y \(c_g\) tal que
\begin{itemize}
\item para toda \(n \geq n_g\), \(c_g h(n) \geq g(n)\)
\end{itemize}
\end{itemize}

\noindent
Entonces:
$$
c_f c_g h(n) \geq f(n) \quad \mbox{para toda} \quad n \geq \max\{n_f, n_g\}
$$
\subsubsection{Si \(f(n)\) es \(\Theta(g(n))\) y \(g(n)\) es \(\Theta(h(n))\) entonces \(f(n)\) es \(\Theta(h(n))\)}
\label{sec:org588fd80}
Esta comprobación es una combinación de las pruebas de transitividad para \(O(n)\)
y \(\Omega(n)\). Asumiendo que el antecedente es cierto:
\begin{itemize}
\item Si \(f(n)\) es \(\Theta(g(n))\), existen las constantes \(c_f, c_f'\) y \(n_f\) tal que
\begin{itemize}
\item \(c_f' g(n) \leq f(n) \leq c_f g(n)\) para toda \(n \geq n_f\)
\end{itemize}
\item Si \(g(n)\) es \(\Theta(h(n))\), existen las constantes \(c_g, c_g'\) y \(n_g\) tal que
\begin{itemize}
\item \(c_g' h(n) \leq g(n) \leq c_g h(n)\) para toda \(n \geq n_g\)
\end{itemize}
\end{itemize}
\parg
Tenemos entonces:
$$
c_f' g(n) \leq f(n) \quad \mbox{y} \quad c_g' h(n) \leq g(n)
$$
si multiplicamos la ecuación de la derecha por la \textbf{constante positiva} \(c_f'\),
obtenemos:
$$
c_f' c_g' h(n) \leq c_f' g(n) \leq f(n) \quad \mbox{y por lo tanto} \quad
c_f' c_g' h(n) \leq f(n)
$$
\parg
\noindent
Por otro lado, tenemos:
$$
f(n) \leq  c_f g(n) \quad \mbox{y} \quad g(n) \leq c_g h(n)
$$
si multiplicamos la ecuación de la derecha por la \textbf{constante positiva} \(c_f\),
obtenemos:
$$
f(n) \leq c_f g(n) \leq c_f c_g h(n) \quad \mbox{y por lo tanto} \quad
f(n) \leq c_f c_g h(n)
$$
\parg

\noindent
Si unimos ambas partes, queda:
$$
c_f' c_g' h(n) \leq f(n) \leq c_f c_g h(n) \quad \mbox{para toda} \quad
n \geq \max\{n_f, n_g\}
$$
\subsubsection{Si \(f(n)\) es \(\Omega(g(n))\) y \(g(n)\) es \(\Omega(h(n))\) entonces \(f(n)\) es \(\Omega(h(n))\)}
\label{sec:org169c466}
Debemos asumir que el antecedente es cierto, entonces:
\begin{itemize}
\item Si \(f(n)\) es \(\Omega(g(n))\), existen dos constantes \(n_f'\) y \(c_f'\) tal que
\begin{itemize}
\item para toda \(n \geq n_f'\), \(c_f' g(n) \leq f(n)\)
\end{itemize}
\item Si \(g(n)\) es \(\Omega(h(n))\), existen dos constantes \(n_g'\) y \(c_g'\) tal que
\begin{itemize}
\item para toda \(n \geq n_g'\), \(c_g' h(n) \leq g(n)\)
\end{itemize}
\end{itemize}

Tenemos:
$$
c_f' g(n) \leq f(n)
$$
y
$$
c_g' h(n) \leq g(n)
$$
que al multiplicarse por la constante \textbf{positiva} \(c_f'\) queda
$$
c_f' c_g' h(n) \leq c_f' g(n)
$$
por lo que:
$$
c_f' c_g' h(n) \leq c_f' g(n) \leq f(n)
$$
entonces:

$$
c_f' f_g' h(n) \leq f(n) \quad \mbox{ para toda } \quad n \geq \max\{n_f', n_g'\}
$$
\subsection{Reflexividad}
\label{sec:orgcf4c494}
\subsubsection{Si \(f(n)\) es \(O(f(n))\) entonces \(f(n)\) es \(O(f(n))\)}
\label{sec:org57fdc4b}
\begin{itemize}
\item Si \(f(n)\) es \(O(f(n))\), existen dos constantes \(c_f\) y \(n_f\) tal que
\begin{itemize}
\item \(f(n) \leq c_f f(n)\) para toda \(n \geq n_f\)
\end{itemize}
\end{itemize}

\noindent
Si hacemos \(c_f = 1\), nos queda:
$$
f(n) \leq 1 \cdot f(n) \quad \mbox{para toda} \quad n \geq n_f
$$
o
$$
f(n) \leq f(n) \quad \mbox{para toda} \quad n \geq n_f
$$


\noindent
Con lo que se cumple la condición requerida, ya que \(f(n) = f(n)\).


\noindent
Si, por otro lado, despejamos \(c_f\) de \(f(n) \leq c_f f(n)\), tenemos:
$$
\frac{f(n)}{f(n)} \leq c_f \quad \therefore \quad 1 \leq c_f
$$
de donde se observa que la condición se cumple para toda \(c_f \geq 1\).
\subsubsection{Si \(f(n)\) es \(\Omega(f(n))\) entonces \(f(n)\) es \(\Omega(f(n))\)}
\label{sec:org9cb5b2e}
\begin{itemize}
\item Si \(f(n)\) es \(\Omega(f(n))\), existen dos constantes \(c_f'\) y \(n_f'\) tal que
\begin{itemize}
\item \(c_f' f(n) \leq f(n)\) para toda \(n \geq n_f'\)
\end{itemize}
\end{itemize}

\noindent
Si de igual manera, hacemos \(c_f' = 1\) nos queda:
$$
1 \cdot f(n) \leq f(n) \quad \mbox{para toda} \quad n
$$
o
$$
f(n) \leq f(n) \quad \mbox{para toda} \quad n
$$

\noindent
Con lo que se cumple la condición requerida, ya que \(f(n) = f(n)\).

\noindent
Si, por otro lado, despejamos \(c_f'\) de \(c_f' f(n) \leq f(n)\), tenemos:
$$
c_f' \leq \frac{f(n)}{f(n)} \quad \therefore \quad c_f' \leq 1
$$
de donde se observa que la condición se cumple para toda \(c_f' \leq 1\). Es
importante recordar que en el análisis de los órdenes asintóticos las constantes
se asumen positivas, por lo que en este caso \(0 < c_f' \leq 1\).
\subsubsection{Si \(f(n)\) es \(\Theta(f(n))\) entonces \(f(n)\) es \(\Theta(f(n))\)}
\label{sec:orgdb0c2ff}
\begin{itemize}
\item Si \(f(n)\) es \(\Theta(f(n))\), existen las constantes \(c_f, c_f'\) y \(n_f\) tal
que \(c_f' f(n) \leq f(n) \leq c_f f(n)\)
\end{itemize}

\noindent
Si \(c_f = c_f' = 1\), tenemos:
$$
1 \cdot f(n) \leq f(n) \leq 1 \cdot f(n) \quad \mbox{para toda} \quad n
$$
o
$$
f(n) \leq f(n) \leq f(n) \quad \mbox{para toda} \quad n
$$
\subsection{Simetría}
\label{sec:org389def5}
\subsubsection{\(f(n)\) es \(\Theta(g(n))\) si y sólo si \(g(n)\) es \(\Theta(f(n))\)}
\label{sec:org792f820}
Tomando como cierto el antecedente, tenemos:
\begin{itemize}
\item Si \(f(n)\) es \(\Theta(g(n))\), existen las constantes \(c_f', c_f\) y \(n_f\) tal
que \(c_f' g(n) \leq f(n) \leq c_f g(n)\) para toda \(n \geq n_f\)
\end{itemize}

\noindent
Si a esta desigualdad, la separamos en dos
$$
c_f' g(n) \leq f(n) \quad \mbox{y} \quad f(n) \leq c_f g(n)
$$
si a su vez, a estas desigualdades las manipulamos algebráicamente de modo que
despejemos a \(g(n)\), nos quedan:
$$
g(n) \leq \frac{1}{c_f'} f(n) \quad \mbox{y} \quad \frac{1}{c_f} f(n) \leq g(n)
$$
de donde se infiere que \(\frac{1}{c_f'}\) y \(\frac{1}{c_f}\) son constantes
positivas, con lo que obtenemos
$$
\frac{1}{c_f} f(n) \leq g(n) \leq \frac{1}{c_f'} f(n)
$$

\noindent
Esto cumple los requerimientos para que \(g(n)\) sea \(O(f(n))\).
\subsection{Simetría Transpuesta}
\label{sec:orgc1f3492}
\subsubsection{\(f(n)\) es \(O(g(n))\) si y sólo si \(g(n)\) es \(\Omega(f(n))\)}
\label{sec:org4b4a0c3}
\begin{itemize}
\item Si \(f(n)\) es \(O(g(n))\), existen las constantes \(c_f\) y \(n_f\) tal que
\begin{itemize}
\item \(f(n) \leq c_f g(n)\) para toda \(n \geq n_f\)
\end{itemize}
\end{itemize}

\noindent
Si en dicha desigualdad despejamos \(g(n)\), tenemos:
$$
\frac{1}{c_f} f(n) \leq g(n) \quad \mbox{para toda} \quad n \geq n_f
$$
se observa entonces, por definición, que \(g(n)\) es \(\Omega(g(n))\).
\subsection{Aditividad}
\label{sec:orgc3c608e}
\subsubsection{Si \(f(n)\) es \(O(h(n))\) y \(g(n)\) es \(O(h(n))\), entonces \(f(n)+g(n)\) es \(O(h(n))\)}
\label{sec:orga56e7a1}
\begin{itemize}
\item Si \(f(n)\) es \(O(h(n))\), existen las constantes \(c_f\) y \(n_f\) tal que
\begin{itemize}
\item \(f(n) \leq c_f h(n)\) para toda \(n \geq n_f\)
\end{itemize}
\item Si \(g(n)\) es \(O(h(n))\), existen las constantes \(c_g\) y \(n_g\) tal que
\begin{itemize}
\item \(g(n) \leq c_g h(n)\) para toda \(n \geq n_g\)
\end{itemize}
\end{itemize}

\noindent
Si sumamos ambas desigualdades, se obtiene:
$$
f(n) + g(n) \leq c_f h(n) + c_g h(n) \quad \mbox{para toda} \quad
n \geq \max{n_f, n_g}
$$
que al manipular algebráicamente queda:
$$
f(n) + g(n) \leq (c_f + c_g) h(n) \quad \mbox{para toda} \quad
n \geq \max\{n_f, n_g\}
$$

\noindent
Por definición, se comprueba que \(f(n) + g(n)\) es \(O(h(n))\)
\subsubsection{Si \(f(n)\) es \(\Omega(h(n))\) y \(g(n)\) es \(\Omega(h(n))\), entonces \(f(n)+g(n)\) es \(\Omega(h(n))\)}
\label{sec:org4fdfb90}
\begin{itemize}
\item Si \(f(n)\) es \(\Omega(h(n))\), existen las constantes \(c_f'\) y \(n_f'\) tal que
\begin{itemize}
\item \(c_f' h(n) \leq f(n)\) para toda \(n \geq n_f'\)
\end{itemize}
\item Si \(g(n)\) es \(\Omega(h(n))\), existen las constantes \(c_g'\) y \(n_g'\) tal que
\begin{itemize}
\item \(c_g' h(n) \leq g(n)\) para toda \(n \geq n_g'\)
\end{itemize}
\end{itemize}

\noindent
Si sumamos ambas desigualdades, obtenemos:
$$
c_f' h(n) + c_g' h(n) \leq f(n) + g(n) \quad \mbox{para toda}
    n \geq \max\{n_f', n_g'\}
$$
que al manipular algebráicamente queda:
$$
(c_f' + c_g') h(n) \leq f(n) + g(n) \quad \mbox{para toda} n \geq \max\{n_f', n_g'\}
$$

\noindent
Por definición, se comprueba que \(f(n) + g(n)\) es \(\Omega(h(n))\)
\subsubsection{Si \(f(n)\) es \(\Theta(h(n))\) y \(g(n)\) es \(\Theta(h(n))\), entonces \(f(n)+g(n)\) es \(\Theta(h(n))\)}
\label{sec:org0ddcad2}
\begin{itemize}
\item Si \(f(n)\) es \(\Theta(h(n))\), existe las constantes \(c_f', c_f\) y \(n_f\) tal que
\begin{itemize}
\item \(c_f' h(n) \leq f(n) \leq c_f h(n)\) para toda \(n \geq n_f\)
\end{itemize}
\item Si \(g(n)\) es \(\Theta(h(n))\), existe las constantes \(c_g', c_g\) y \(n_g\) tal que
\begin{itemize}
\item \(c_g' h(n) \leq g(n) \leq c_g h(n)\) para toda \(n \geq n_g\)
\end{itemize}
\end{itemize}

\noindent
Si sumamos ambas desigualdades tenemos:
$$
c_f' h(n) + c_g' h(n) \leq f(n) + g(n) \leq c_f h(n) + c_g h(n)
    \quad \mbox{para toda} \quad
    n \geq \max\{n_f, n_g\}
$$
si a su vez, esta nueva desigualdad la manipulamos algebraicamente, nos queda
$$
(c_f' + c_g') h(n) \leq f(n) + g(n) \leq (c_f + c_h) h(n)
    \quad \mbox{para toda} \quad
    n \geq \max\{n_f, n_g\}
$$

\noindent
Por definición, se comprueba que \(f(n) + g(n)\) es \(\Theta(h(n))\)
\end{document}
